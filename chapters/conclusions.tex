\chapter{Conclusions and future work}

% TODO closing thoughts
% future work
% - more work on delays, e.g.
% 11 key size, MACK section length and MAC size are all provided in the
%    initial blocks of a DSM-KROOT. These are needed by the receiver in order to be
%    able to successfully decode the MACK sections. The problem here is that the
%    DSM-KROOT is transmitted at a way lower rate than the MACK section. It would
%    be interesting to:
%     - describe the worst case delay until a receiver can start authenticate
%       data
%     - provide a suggestion on how a receiver should implement different
%       stages: one in which it receives the operational parameters, one in
%       which it can receive/decode in parallel a DSM-KROOT and the MACKs, and
%       ome in which it can also authenticate the data
% - answers are needed to understand how to react to security problems: should
%   th receiver be strict or not?
% - simulation of the security protocol will help giving more guidance to
%   receiver implementors
% - security of the overall protocol is left in the hands of implementors: for
%   example the protocol doesn't enforce the initial clock synchronization, so
%   poor implementation might expose receivers to attacks with a false sense of
%   security. Would be interesting to see work in the direction of enforcing
%   certain practices

\section{Conclusions}
While analyzing the OSNMA protocol for Galileo, a few distinct points became
clear. This implementation of the TESLA protocol had a hard challenge in front:
adding a layer of security on top of a protocol that is unidirectional with no
guarantee of bi-directional communication possibilities and with an already low
transmission rate. All of this required an effort that resulted in some
adaptations of the original security scheme in order to support aspects like the
presence of multiple transmitters, which weren't part of the design requirements
of TESLA.

All of this carries along more complexity on the side of the receiver, which
needs to be able to handle a more elaborated state machine and handle more
possible error states than before. Another trend that emerges from the analysis
of the computational requirements for the protocol is in general the
availability of more computational resources to perform a high number of
cryptographic operations in a short period of time. At the same time, this
requirement comes along with the need of not increasing the amount of power
consumed in performing authentication of the navigation message. In other words,
the need for security and flexibility of OSNMA needs to be traded off with the
need of keeping the energy profile of the receiver compelling for portable use.
At the same time, it has been clearly exposed how higher quality local clocks
might also be required if security of radio navigation is to become a commodity;
the lower the quality, the higher the chances that an individual receiver might
fall out of the acceptable range of clock drift.

\vspace{\baselineskip}

This panorama suggests two opposite points of views, which at the same time
don't entirely contradict each other. On one side, one could try to minimize the
impact of this security implementation on the current design of hardware
embedded receivers. This is doable by being clear on the minimum security
requirements (for example, in terms of clock synchronization), and by having
some support on the transmitter side on the strategy for the transmission of
secure information (for example, the amount and frequency of transmitted
floating KROOTs).

On the other side, one could take the introduction of such protocols as a sign
of times to come, which speak of the need for increased receiver complexity and
computational power. From this point of view, the power is in the hands of the
implementor to find innovative ways to build receivers capable of complying to
more demanding applications while retaining their appeal to the market.

\section{Future work}
