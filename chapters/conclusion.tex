\chapter{Conclusion}
\label{ch:conclusion}

\section{Future work}

During the analysis a few aspects were highlighted that could benefit from
future research. The first of them is related to the fact that even in an ideal
situation where no errors are introduced in the transmission of data,
authentication of the navigation message takes a time in the order of minutes.
This is a constraint that might be too stringent for certain applications, and
it would be interesting to understand what possibilities the current message
structure offers in order to improve performances of a first authenticated
position fix.

\vspace{\baselineskip}

Another topic that requires some discussion is related to the error conditions
in the protocol. Generally from the analysis a tradeoff emerged: either security
is mandatory or it's considered as a mere "upgrade" to the unauthenticated
service. Depending on the perspective taken, treatment of error conditions can
follow very different paths. In the first case, with an extreme approach the
receiver interprets every kind of error as an attempt to spoof the data, which
requires as a countermeasure a complete discard of the information received from
a particular satellite. In the second case, the receiver still identifies the
security breach but realizes that a complete shutdown of the navigation
operation is not possible. In that case, the receiver could either downgrade
communication to the default non-secure protocol, or it could put the choice in
the hands of the user (one example of this approach are browsers that fail to
authenticate a server exposed through an HTTPS endpoint). Both of these choices
offer advantages and drawbacks. A complete discussion of those could provide
clearer guidelines to receiver implementors, and eventually identify more
sophisticated approaches to identify real security breaches and recoverable
scenarios.

\vspace{\baselineskip}

To facilitate the analysis of these and other aspects, a software simulator
could be of great benefit. As of the time of writing, several simulators exist
for GPS and to a lesser extent for Galileo - both on the receiver and on the
transmitter side. Nevertheless, none of them yet implements OSNMA. A project
that implements it would be of aid in performing more in-depth analysis of the
attack patterns and the potential implementation problems that hardware
implementors might face.

\vspace{\baselineskip}

Finally, one overarching aspect emerged in most of the point touched by the
analysis conducted in this document. It is clear that the security of the
overall protocol, as it is today, relies heavily on implementors to follow the
security guidelines in order for it to be effective against the attacks it
promises to block. As an example of this, let's consider clock synchronization:
should receivers ignore or miscalculate the drift threshold, the receiver would
be exposed to spoofing attacks that might be even subtler to identify as the
authentication layer might give a false sense of security. This aspect is even
more relevant considering that, for example, in the case of clock
synchronization the receiver requires certain external conditions to be met (for
example, that the device doesn't remain inactive for too long a period). While
in practice this might not be a problem in a high percentage of cases, it is
still an architectural constraint that might consitute a roadblock for certain
applications. From this point of view, one way to prevent this to happen is to
perform research on ways to overcome the external constraint by either using
external aid in bringing the receiver in a state that matches the security
conditions, or by working on an expansion of the protocol that could remove the
constraint altogether. One idea could be for example to provide some form of
authenticated time the receiver can use as a starting point to close its gap in
clock synchronization. The time could be provided in a fashion similar to that
of the DSM-KROOT, at a slower pace but authenticated in a way that's infeasible
to tamper.

\section{Final words}

While analyzing the OSNMA protocol for Galileo, a few distinct points became
clear. This implementation of the TESLA protocol had a hard challenge in front:
adding a layer of security on top of a protocol that is unidirectional with no
guarantee of bi-directional communication possibilities and with an already low
transmission rate. All of this required an effort that resulted in some
adaptations of the original security scheme in order to support aspects like the
presence of multiple transmitters, which weren't part of the design requirements
of TESLA.

All of this carries along more complexity on the side of the receiver, which
needs to be able to handle a more elaborated state machine and handle more
possible error states than before. Another trend that emerges from the analysis
of the computational requirements for the protocol is in general the
availability of more computational resources to perform a high number of
cryptographic operations in a short period of time. At the same time, this
requirement comes along with the need of not increasing the amount of power
consumed in performing authentication of the navigation message. In other words,
the need for security and flexibility of OSNMA needs to be traded off with the
need of keeping the energy profile of the receiver compelling for portable use.
At the same time, it has been clearly exposed how higher quality local clocks
might also be required if security of radio navigation is to become a commodity;
the lower the quality, the higher the chances that an individual receiver might
fall out of the acceptable range of clock drift.

\vspace{\baselineskip}

This panorama suggests two opposite points of views, which at the same time
don't entirely contradict each other. On one side, one could try to minimize the
impact of this security implementation on the current design of hardware
embedded receivers. This is doable by being clear on the minimum security
requirements (for example, in terms of clock synchronization), and by having
some support on the transmitter side on the strategy for the transmission of
secure information (for example, the amount and frequency of transmitted
floating KROOTs).

On the other side, one could take the introduction of such protocols as a sign
of times to come, which speak of the need for increased receiver complexity and
computational power. From this point of view, the power is in the hands of the
implementor to find innovative ways to build receivers capable of complying to
more demanding applications while retaining their appeal to the market.
