%!TEX root = ../dissertation.tex

% outline
% - theoretical framework (6)
%   - functioning principle and requirements: TOA, triangulation, PVT, precise
%   timing (1)
%   - obtaining user to satellite range (2)
%   - obtaining user position (2)
%   - obtaining user velocity (1)
% - practical aspects (4)
%   - measuring time
%   - determining satellite position
%   - earth reference frame
% - how a receiver works (?)

\chapter{Satellite navigation fundamentals}

No matter the GNSS under analysis, the fundamental principles that make radio
satellite navigation work are the same. In this chapter we'll describe the
equations that make satellite navigation a reality; understanding these
mechanics will be of importance when talking about the different ways of
authenticating a navigation signal.

\section{Overview of ranging through TOA measurements}
To introduce the problem of satellite navigation, let's assume a generic actor
at position $P$ wanting to know its position in a three-dimensional coordinate
frame. Let's assume also the presence of a beacon, whose position $T$ is fixed
and precisely known. This beacon transmits messages at regular intervals, which
the actor is capable of intercepting. These messages contain two pieces of
information: the precise time $t_0$ at which the message was sent, and a
description of $T$, i.e. the position of the beacon at the time of transmission.
Let's also assume that the transmission happens over a medium with a specific
speed of propagation $s$ determined by the law of physics. Finally, let's assume
both actor and beacon are equipped with perfectly synchronized clocks, so that
locally observed time for each party is equal to that of the other counterpart.

Given these assumptions the actor has all the information needed to determine
its position with respect to the beacon. The resolution of this problem is based
on the Time Of Arrival (TOA) of the message. If the actor marks the time $t_1$
at which the message arrives, then its distance from the beacon can be computed
by multiplying the flight time of the message $t_1 - t_0$ by its travelling
speed $s$. The resulting range measurement $r_1$ defines a sphere of radius
$r_1$ centered at the position of the beacon $T$, and the actor position is a
point of this sphere.

To further narrow down the precise location of the actor $P$, it's sufficient to
have another two beacons at positions different than $T$. With the same
technique, the actor can determine its distance from the other two beacons.
Determining its distance $r_2$ from the second beacon will mean its position is
on a sphere of radius $r_2$ centered around the beacon. Intersecting this sphere
with the sphere around the first beacon will yield a circumference (or, in its
degenerate case, a point if the two spheres are tangent - a condition that will
be ignored for now as it's not the general case). Repeating the measurement with
the third beacon will also again yield a sphere of radius $r_3$ that can be
intersected with the other two, resulting in the identification of two points,
one being the actor's true position and the other being its mirror with respect
to the beacons' plane. In order to uniquely determine its position, the actor
can either perform a fourth measurement, or introduce other constraints in the
equation that automatically exclude one of the two points. For example, for
users on the Earth's surface, the point with lower altitude can be selected with
no further measurement.

\par

This contrived example forms the basic of triangulation through TOA measurement,
which is at the base of GNSS. In the real world, though, the hypotesis that
allowed the problem to be solved in such simple terms is never true, and using
this simplified set of constraints is not enough to provide precise positioning.
Therefore, more relaxed constraints and more complex models are used as a basis
for satellite navigation, which we'll describe next. Nevertheless, the
fundamental concept remains unchanged even with a more elaborated model of the
world.

\section{Determining satellite-to-user range}
Improving on the model just described, the first assumption we'll refine is that
of the clock of the receiver being perfectly synchronzied with that of the
transmitter.
