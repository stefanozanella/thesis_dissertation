%!TEX root = ../dissertation.tex

% outline
% - theoretical framework (6)
%   - functioning principle and requirements: TOA, triangulation, PVT, precise
%   timing (1)
%   - obtaining user to satellite range (2)
%   - obtaining user position (2)
%   - obtaining user velocity (1)
% - practical aspects (4)
%   - measuring time
%   - determining satellite position
%   - earth reference frame
% - how a receiver works (?)

\chapter{Satellite navigation fundamentals}

No matter the GNSS under analysis, the fundamental principles that make radio
satellite navigation work are the same. In this chapter we'll describe the
equations that make satellite navigation a reality; understanding these
mechanics will be of importance when talking about the different ways of
authenticating a navigation signal.

\section{Overview of ranging through TOA measurements}
To introduce the problem of satellite navigation, let's assume a generic actor
at position $P$ wanting to know its position in a three-dimensional coordinate
frame. Let's assume also the presence of a beacon, whose position $T$ is fixed
and precisely known. This beacon transmits messages at regular intervals, which
the actor is capable of intercepting. These messages contain two pieces of
information: the precise time $t_0$ at which the message was sent, and a
description of $T$, i.e. the position of the beacon at the time of transmission.
Let's also assume that the transmission happens over a medium with a specific
speed of propagation $s$ determined by the law of physics. Finally, let's assume
both actor and beacon are equipped with perfectly synchronized clocks, so that
locally observed time for each party is equal to that of the other counterpart.

Given these assumptions the actor has all the information needed to determine
its position with respect to the beacon. The resolution of this problem is based
on the Time Of Arrival (TOA) of the message. If the actor marks the time $t_1$
at which the message arrives, then its distance from the beacon can be computed
by multiplying the flight time of the message $t_1 - t_0$ by its travelling
speed $s$. The resulting range measurement $r_1$ defines a sphere of radius
$r_1$ centered at the position of the beacon $T$, and the actor position is a
point of this sphere.

To further narrow down the precise location of the actor $P$, it's sufficient to
have another two beacons at positions different than $T$. With the same
technique, the actor can determine its distance from the other two beacons.
Determining its distance $r_2$ from the second beacon will mean its position is
on a sphere of radius $r_2$ centered around the beacon. Intersecting this sphere
with the sphere around the first beacon will yield a circumference (or, in its
degenerate case, a point if the two spheres are tangent - a condition that will
be ignored for now as it's not the general case). Repeating the measurement with
the third beacon will also again yield a sphere of radius $r_3$ that can be
intersected with the other two, resulting in the identification of two points,
one being the actor's true position and the other being its mirror with respect
to the beacons' plane. In order to uniquely determine its position, the actor
can either perform a fourth measurement, or introduce other constraints in the
equation that automatically exclude one of the two points. For example, for
users on the Earth's surface, the point with lower altitude can be selected with
no further measurement.

\par

This contrived example forms the basic of triangulation through TOA measurement,
which is at the base of GNSS. In the real world, though, the hypotesis that
allowed the problem to be solved in such simple terms is never true, and using
this simplified set of constraints is not enough to provide precise positioning.
Therefore, more relaxed constraints and more complex models are used as a basis
for satellite navigation, which we'll describe next. Nevertheless, the
fundamental concept remains unchanged even with a more elaborated model of the
world.

\section{Determining satellite-to-user range}
Improving on the model just described, the first assumption we'll refine is that
of the clock of the receiver being perfectly synchronzied with that of the
transmitter. In the model for PVT calculation there are many factors that can
influence the final result, but hardly any of those can have an impact as big as
that of an imprecise time calculation.

\par

Before talking about how unsynchronized clocks affect the position resolution,
let's define the general model used by GNSS.

As saw already, the model for PVT calculation includes two clocks: that of the
transmitter and that of the receiver. With differences in precision and amount
of drift, both clocks are assumed to accumulate (positive or negative) drift
with respect to system time, thus being offset from the reference time.
We'll call the offset of the transmitter's clock $\delta_t$ and the offset at
the receiver's $\delta_r$.

Recalling that the geometric distance $r$ between transmitter and receiver is
\begin{equation}
  r = c(T_r - T_t)
\end{equation}
where
\begin{itemize}
  \item $c$ is the speed of light at which the signal travels
  \item $T_r$ is the system time of signal reception
  \item $T_t$ is the system time of signal transmission
\end{itemize}
we can then improve the equation by including the time offsets just described,
which gives us the equation for the \textit{pseudorange} $\rho$:
\begin{equation}
  \begin{aligned}
    \rho &= c[(T_r + \delta_r) - (T_t + \delta_t)] \\
    & = c[(T_r + \delta_r) - (T_t + \delta_t)] \\
    & = c(T_r - T_t) + c(\delta_r - \delta_t) \\
    & = r + c(\delta_r - \delta_t)
  \end{aligned}
\end{equation}

Of these two offset, that of the satellite is monitored and controlled by each
GNSS ground mission, and its correction transmitted as part of the navigation
message. Therefore, despite some residual offset its contribution to the
equation can be ignored, thus simplifying the model for the pseudorange to
\begin{equation} \label{eq:pseudorange}
  \rho = r + c(\delta_r)
\end{equation}

This transforms the original problem from having three variables to having four.

\section{Determining user's position}
Solution of equation \ref{eq:pseudorange} yields the magnitude of a vector that
represents the distance between transmitter and receiver. In reality, what we're
interested in is identifying the coordinates of the receiver within a reference
frame. To model this we can expand transforming $r$ in cartesian form. Given a
receiver at position $(x_r, y_r, z_r)$ and a transmitter at position $(x_t, y_t,
z_t)$, the pseudorange can be expressed as:
\begin{equation} \label{eq:expandedpseudorange}
  \begin{aligned}
    \rho &= \sqrt{(x_t - x_r)^2 + (y_t - y_r)^2 + (z_t - z_r)^2} + c\delta_r \\
    & = f(x_r, y_r, z_r, \delta_r)
  \end{aligned}
\end{equation}

Given the problem contains four variables, we need a set of four equations to
obtained a closed solution. This is equivalent of obtaining ranges from four
different satellites, which can be expressed with the set of equations

\begin{equation} \label{eq:userpositionfull}
  \begin{array}{l}
  \rho_1 = \sqrt{(x_{t1} - x_r)^2 + (y_{t1} - y_r)^2 + (z_{t1} - z_r)^2} + c\delta_r \\
  \rho_2 = \sqrt{(x_{t2} - x_r)^2 + (y_{t2} - y_r)^2 + (z_{t2} - z_r)^2} + c\delta_r \\
  \rho_3 = \sqrt{(x_{t3} - x_r)^2 + (y_{t3} - y_r)^2 + (z_{t3} - z_r)^2} + c\delta_r \\
  \rho_4 = \sqrt{(x_{t4} - x_r)^2 + (y_{t4} - y_r)^2 + (z_{t4} - z_r)^2} + c\delta_r
  \end{array}
\end{equation}

This set of nonlinear equation can be solved with different techniques:
closed-form solutions, iterative methods based on linearization, or Kalman
filters. What follows is an example of linearization.

\par

To transform non-linear terms in the equation, we will make the assumption that
an approximate position $(\hat x_r, \hat y_r, \hat z_r)$ and clock error
estimate $\hat \delta_r$ are available. Under this assumption we can then say
that the true receiver position and time are composed of an approximate
component and an incremental one:
\begin{equation} \label{eq:incrementals}
  \begin{array}{l}
    x_r = \hat x_r + \Delta x_r \\
    y_r = \hat y_r + \Delta y_r \\
    z_r = \hat z_r + \Delta z_r \\
    \delta_r = \hat \delta_r + \Delta \delta_r
  \end{array}
\end{equation}

This allows us to express \ref{eq:expandedpseudorange} as
\begin{equation}
    f(x_r, y_r, z_r, \delta_r) = f(\hat x_r + \Delta x_r, \hat y_r + \Delta y_r,
      \hat z_r + \Delta z_r, \hat \delta_r + \Delta \delta_r)
\end{equation}

This can in turn be expaneded using a Taylor series as
\begin{equation} \label{eq:taylorseries}
  \begin{aligned}
  f(\hat x_r + \Delta x_r, \hat y_r + \Delta y_r,
      \hat z_r + \Delta z_r, \hat \delta_r + \Delta \delta_r) &= f(\hat x_r, \hat y_r,
      \hat z_r, \hat \delta_r) + \\
      \frac{\partial f(\hat x_r, \hat y_r, \hat z_r, \hat \delta_r)}{\partial \hat x_r} \Delta x_r + \\
      \frac{\partial f(\hat x_r, \hat y_r, \hat z_r, \hat \delta_r)}{\partial \hat y_r} \Delta y_r + \\
      \frac{\partial f(\hat x_r, \hat y_r, \hat z_r, \hat \delta_r)}{\partial \hat z_r} \Delta z_r + \\
      \frac{\partial f(\hat x_r, \hat y_r, \hat z_r, \hat \delta_r)}{\partial \hat \delta_r} \Delta \delta_r + \\
      ...
  \end{aligned}
\end{equation}

The expression has been truncated to eliminate higher order non-linear
components. Focusing on the partial derivatives we obtain
\begin{equation} \label{eq:partialsexpansion}
  \begin{array}{l}
    \frac{\partial f(\hat x_r, \hat y_r, \hat z_r, \hat \delta_r)}{\partial \hat x_r} = - \frac{x_t - \hat x_r}{\hat r_t} \\
    \frac{\partial f(\hat x_r, \hat y_r, \hat z_r, \hat \delta_r)}{\partial \hat y_r} = - \frac{y_t - \hat y_r}{\hat r_t} \\
    \frac{\partial f(\hat x_r, \hat y_r, \hat z_r, \hat \delta_r)}{\partial \hat z_r} = - \frac{z_t - \hat z_r}{\hat r_t} \\
    \frac{\partial f(\hat x_r, \hat y_r, \hat z_r, \hat \delta_r)}{\partial \hat \delta_r} = c
  \end{array}
\end{equation}

where
\begin{equation}
  \hat r_t = \sqrt{(x_t - \hat x_r)^2 + (y_t - \hat y_r)^2 + (z_t - \hat z_r)^2}
\end{equation}

At this point we can substitute \ref{eq:expandedpseudorange} and
\ref{eq:partialsexpansion} into \ref{eq:taylorseries}, which yields (for a
generic range between transmitter $j$ and receiver $r$):
\begin{equation}
  \rho_j = \hat \rho_j - \frac{x_j - \hat x_r}{\hat r_j}\Delta x_r - \frac{y_j -
    \hat y_r}{\hat r_j}\Delta y_r - \frac{z_j - \hat z_r}{\hat r_j}\Delta z_r +
    c\delta_r
\end{equation}

This completes the linearization of \ref{eq:expandedpseudorange} with respect to
the unknown $\Delta x_r, \Delta y_r, \Delta z_r, \Delta \delta_r$. For clarity
we can rearrange unknowns on the right and knowns on the left
\begin{equation}
  \hat \rho_j - \rho_j = \frac{x_j - \hat x_r}{\hat r_j}\Delta x_r + \frac{y_j -
    \hat y_r}{\hat r_j}\Delta y_r + \frac{z_j - \hat z_r}{\hat r_j}\Delta z_r -
    c\delta_r
\end{equation}
and then introduce new variables
\begin{equation}
  \begin{array}{l}
    \Delta \rho = \hat \rho_j - \rho_j \\
    a_{xj} = \frac{x_j - \hat x_r}{\hat r_j} \\
    a_{yj} = \frac{y_j - \hat x_r}{\hat r_j} \\
    a_{zj} = \frac{z_j - \hat z_r}{\hat r_j}
  \end{array}
\end{equation}
which allows us to rewrite the equation as
\begin{equation}
  \Delta \rho_j = a_{xj} \Delta x_r + a_{yj} \Delta y_r + a_{zj} \Delta z_r - c \delta_r
\end{equation}

Now the original formulation of the problem stated in \ref{eq:userpositionfull}
can be rewritten in linear form as
\begin{equation}
  \begin{array}{l}
    \Delta \rho_1 = a_{x1} \Delta x_r + a_{y1} \Delta y_r + a_{z1} \Delta z_r - c \delta_r \\
    \Delta \rho_2 = a_{x2} \Delta x_r + a_{y2} \Delta y_r + a_{z2} \Delta z_r - c \delta_r \\
    \Delta \rho_3 = a_{x3} \Delta x_r + a_{y3} \Delta y_r + a_{z3} \Delta z_r - c \delta_r \\
    \Delta \rho_4 = a_{x4} \Delta x_r + a_{y4} \Delta y_r + a_{z4} \Delta z_r - c \delta_r
  \end{array}
\end{equation}

These equations can then be put in matrix form; by defining
\[
  \begin{array}{l}
  \Delta \bm{\rho} = \begin{bmatrix}
    \Delta \rho_1 \\
    \Delta \rho_2 \\
    \Delta \rho_3 \\
    \Delta \rho_4
  \end{bmatrix}

  \>

  \bm{H} = \begin{bmatrix}
    a_{x1} & a_{y1} & a_{z1} & 1 \\
    a_{x2} & a_{y2} & a_{z2} & 1 \\
    a_{x3} & a_{y3} & a_{z3} & 1 \\
    a_{x4} & a_{y4} & a_{z4} & 1 \\
  \end{bmatrix}

  \>

  \Delta \bm{x} = \begin{bmatrix}
    \Delta x_r \\
    \Delta y_r \\
    \Delta z_r \\
    -c \Delta \delta_r
  \end{bmatrix}
  \end{array}
\]
we can then write
\begin{equation}
  \Delta \bm{\rho} = \bm{H} \Delta \bm{x}
\end{equation}
which has solution
\begin{equation}
  \Delta \bm{x} = \bm{H}^{-1} \Delta \bm{\rho}
\end{equation}

Once the unknown are computed, the actual user position can be computed using
\ref{eq:incrementals}. This process works well as long as the displacement from
the approximate position is small enough. However, subsequent iterations of the
method will improve the accuracy of the result.

In reality though, the true user-to-satellite range measurements are affected by
independent errors such as measurement noise and multipath. Such errors
constitute a component that affects the calculated user position. To minimize
these errors, one technique is to perform measurements to more than four
satellites, using least square estimation methods to minimize the effect of the
independent source of errors. This is a feature that most of today's user
receivers employ.
