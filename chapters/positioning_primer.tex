%!TEX root = ../dissertation.tex

\chapter{Satellite navigation fundamentals}

No matter the GNSS under analysis, the fundamental principles that make radio
satellite navigation work are the same. In this chapter we'll describe the
equations that make satellite navigation a reality; understanding these
mechanics will be of importance when talking about the different ways of
authenticating a navigation signal.

\section{Overview of ranging through TOA measurements}
To introduce the problem of satellite navigation, let's assume a generic actor
at position $P$ wanting to know its position in a three-dimensional coordinate
frame. Let's assume also the presence of a beacon, whose position $T$ is fixed
and precisely known. This beacon transmits messages at regular intervals, which
the actor is capable of intercepting. These messages contain two pieces of
information: the precise time $t_0$ at which the message was sent, and a
description of $T$, i.e. the position of the beacon at the time of transmission.
Let's also assume that the transmission happens over a medium with a specific
speed of propagation $s$ determined by the law of physics. Finally, let's assume
both actor and beacon are equipped with perfectly synchronized clocks, so that
locally observed time for each party is equal to that of the other counterpart.

Given these assumptions the actor has all the information needed to determine
its position with respect to the beacon. The resolution of this problem is based
on the Time Of Arrival (TOA) of the message. If the actor marks the time $t_1$
at which the message arrives, then its distance from the beacon can be computed
by multiplying the flight time of the message $t_1 - t_0$ by its travelling
speed $s$. The resulting range measurement $r_1$ defines a sphere of radius
$r_1$ centered at the position of the beacon $T$, and the actor position is a
point of this sphere.

To further narrow down the precise location of the actor $P$, it's sufficient to
have another two beacons at positions different than $T$. With the same
technique, the actor can determine its distance from the other two beacons.
Determining its distance $r_2$ from the second beacon will mean its position is
on a sphere of radius $r_2$ centered around the beacon. Intersecting this sphere
with the sphere around the first beacon will yield a circumference (or, in its
degenerate case, a point if the two spheres are tangent - a condition that will
be ignored for now as it's not the general case). Repeating the measurement with
the third beacon will also again yield a sphere of radius $r_3$ that can be
intersected with the other two, resulting in the identification of two points,
one being the actor's true position and the other being its mirror with respect
to the beacons' plane. In order to uniquely determine its position, the actor
can either perform a fourth measurement, or introduce other constraints in the
equation that automatically exclude one of the two points. For example, for
users on the Earth's surface, the point with lower altitude can be selected with
no further measurement.

\par

This contrived example forms the basic of triangulation through TOA measurement,
which is at the base of GNSS. In the real world, though, the hypotesis that
allowed the problem to be solved in such simple terms is never true, and using
this simplified set of constraints is not enough to provide precise positioning.
Therefore, more relaxed constraints and more complex models are used as a basis
for satellite navigation, which we'll describe next. Nevertheless, the
fundamental concept remains unchanged even with a more elaborated model of the
world.

\section{Determining satellite-to-user range}
Improving on the model just described, the first assumption we'll refine is that
of the clock of the receiver being perfectly synchronzied with that of the
transmitter. In the model for PVT calculation there are many factors that can
influence the final result, but hardly any of those can have an impact as big as
that of an imprecise time calculation.

\par

Before talking about how unsynchronized clocks affect the position resolution,
let's define the general model used by GNSS.

As saw already, the model for PVT calculation includes two clocks: that of the
transmitter and that of the receiver. With differences in precision and amount
of drift, both clocks are assumed to accumulate (positive or negative) drift
with respect to system time, thus being offset from the reference time.
We'll call the offset of the transmitter's clock $\delta_t$ and the offset at
the receiver's $\delta_r$.

Recalling that the geometric distance $r$ between transmitter and receiver is
\begin{equation}
  r = c(T_r - T_t)
\end{equation}
where
\begin{itemize}
  \item $c$ is the speed of light at which the signal travels
  \item $T_r$ is the system time of signal reception
  \item $T_t$ is the system time of signal transmission
\end{itemize}
we can then improve the equation by including the time offsets just described,
which gives us the equation for the \textit{pseudorange} $\rho$:
\begin{equation}
  \begin{aligned}
    \rho &= c[(T_r + \delta_r) - (T_t + \delta_t)] \\
    & = c[(T_r + \delta_r) - (T_t + \delta_t)] \\
    & = c(T_r - T_t) + c(\delta_r - \delta_t) \\
    & = r + c(\delta_r - \delta_t)
  \end{aligned}
\end{equation}

Of these two offset, that of the satellite is monitored and controlled by each
GNSS ground mission, and its correction transmitted as part of the navigation
message. Therefore, despite some residual offset its contribution to the
equation can be ignored, thus simplifying the model for the pseudorange to
\begin{equation} \label{eq:pseudorange}
  \rho = r + c(\delta_r)
\end{equation}

This transforms the original problem from having three variables to having four.

\section{Determining user's position}
Solution of equation \ref{eq:pseudorange} yields the magnitude of a vector that
represents the distance between transmitter and receiver. In reality, what we're
interested in is identifying the coordinates of the receiver within a reference
frame. To model this we can expand transforming $r$ in cartesian form. Given a
receiver at position $(x_r, y_r, z_r)$ and a transmitter at position $(x_t, y_t,
z_t)$, the pseudorange can be expressed as:
\begin{equation} \label{eq:expandedpseudorange}
  \begin{aligned}
    \rho &= \sqrt{(x_t - x_r)^2 + (y_t - y_r)^2 + (z_t - z_r)^2} + c\delta_r \\
    & = f(x_r, y_r, z_r, \delta_r)
  \end{aligned}
\end{equation}

Given the problem contains four variables, we need a set of four equations to
obtained a closed solution. This is equivalent of obtaining ranges from four
different satellites, which can be expressed with the set of equations

\begin{equation} \label{eq:userpositionfull}
  \begin{array}{l}
  \rho_1 = \sqrt{(x_{t1} - x_r)^2 + (y_{t1} - y_r)^2 + (z_{t1} - z_r)^2} + c\delta_r \\
  \rho_2 = \sqrt{(x_{t2} - x_r)^2 + (y_{t2} - y_r)^2 + (z_{t2} - z_r)^2} + c\delta_r \\
  \rho_3 = \sqrt{(x_{t3} - x_r)^2 + (y_{t3} - y_r)^2 + (z_{t3} - z_r)^2} + c\delta_r \\
  \rho_4 = \sqrt{(x_{t4} - x_r)^2 + (y_{t4} - y_r)^2 + (z_{t4} - z_r)^2} + c\delta_r
  \end{array}
\end{equation}

This set of nonlinear equation can be solved with different techniques:
closed-form solutions, iterative methods based on linearization, or Kalman
filters. What follows is an example of linearization.

\par

To transform non-linear terms in the equation, we will make the assumption that
an approximate position $(\hat x_r, \hat y_r, \hat z_r)$ and clock error
estimate $\hat \delta_r$ are available. Under this assumption we can then say
that the true receiver position and time are composed of an approximate
component and an incremental one:
\begin{equation} \label{eq:incrementals}
  \begin{array}{l}
    x_r = \hat x_r + \Delta x_r \\
    y_r = \hat y_r + \Delta y_r \\
    z_r = \hat z_r + \Delta z_r \\
    \delta_r = \hat \delta_r + \Delta \delta_r
  \end{array}
\end{equation}

This allows us to express \ref{eq:expandedpseudorange} as
\begin{equation}
    f(x_r, y_r, z_r, \delta_r) = f(\hat x_r + \Delta x_r, \hat y_r + \Delta y_r,
      \hat z_r + \Delta z_r, \hat \delta_r + \Delta \delta_r)
\end{equation}

This can in turn be expaneded using a Taylor series as
\begin{equation} \label{eq:taylorseries}
  \begin{aligned}
  f(\hat x_r + \Delta x_r, \hat y_r + \Delta y_r,
      \hat z_r + \Delta z_r, \hat \delta_r + \Delta \delta_r) &= f(\hat x_r, \hat y_r,
      \hat z_r, \hat \delta_r) + \\
      \frac{\partial f(\hat x_r, \hat y_r, \hat z_r, \hat \delta_r)}{\partial \hat x_r} \Delta x_r + \\
      \frac{\partial f(\hat x_r, \hat y_r, \hat z_r, \hat \delta_r)}{\partial \hat y_r} \Delta y_r + \\
      \frac{\partial f(\hat x_r, \hat y_r, \hat z_r, \hat \delta_r)}{\partial \hat z_r} \Delta z_r + \\
      \frac{\partial f(\hat x_r, \hat y_r, \hat z_r, \hat \delta_r)}{\partial \hat \delta_r} \Delta \delta_r + \\
      ...
  \end{aligned}
\end{equation}

The expression has been truncated to eliminate higher order non-linear
components. Focusing on the partial derivatives we obtain
\begin{equation} \label{eq:partialsexpansion}
  \begin{array}{l}
    \frac{\partial f(\hat x_r, \hat y_r, \hat z_r, \hat \delta_r)}{\partial \hat x_r} = - \frac{x_t - \hat x_r}{\hat r_t} \\
    \frac{\partial f(\hat x_r, \hat y_r, \hat z_r, \hat \delta_r)}{\partial \hat y_r} = - \frac{y_t - \hat y_r}{\hat r_t} \\
    \frac{\partial f(\hat x_r, \hat y_r, \hat z_r, \hat \delta_r)}{\partial \hat z_r} = - \frac{z_t - \hat z_r}{\hat r_t} \\
    \frac{\partial f(\hat x_r, \hat y_r, \hat z_r, \hat \delta_r)}{\partial \hat \delta_r} = c
  \end{array}
\end{equation}

where
\begin{equation}
  \hat r_t = \sqrt{(x_t - \hat x_r)^2 + (y_t - \hat y_r)^2 + (z_t - \hat z_r)^2}
\end{equation}

At this point we can substitute \ref{eq:expandedpseudorange} and
\ref{eq:partialsexpansion} into \ref{eq:taylorseries}, which yields (for a
generic range between transmitter $j$ and receiver $r$):
\begin{equation}
  \rho_j = \hat \rho_j - \frac{x_j - \hat x_r}{\hat r_j}\Delta x_r - \frac{y_j -
    \hat y_r}{\hat r_j}\Delta y_r - \frac{z_j - \hat z_r}{\hat r_j}\Delta z_r +
    c\delta_r
\end{equation}

This completes the linearization of \ref{eq:expandedpseudorange} with respect to
the unknown $\Delta x_r, \Delta y_r, \Delta z_r, \Delta \delta_r$. For clarity
we can rearrange unknowns on the right and knowns on the left
\begin{equation}
  \hat \rho_j - \rho_j = \frac{x_j - \hat x_r}{\hat r_j}\Delta x_r + \frac{y_j -
    \hat y_r}{\hat r_j}\Delta y_r + \frac{z_j - \hat z_r}{\hat r_j}\Delta z_r -
    c\delta_r
\end{equation}
and then introduce new variables
\begin{equation}
  \begin{array}{l}
    \Delta \rho = \hat \rho_j - \rho_j \\
    a_{xj} = \frac{x_j - \hat x_r}{\hat r_j} \\
    a_{yj} = \frac{y_j - \hat x_r}{\hat r_j} \\
    a_{zj} = \frac{z_j - \hat z_r}{\hat r_j}
  \end{array}
\end{equation}
which allows us to rewrite the equation as
\begin{equation}
  \Delta \rho_j = a_{xj} \Delta x_r + a_{yj} \Delta y_r + a_{zj} \Delta z_r - c \delta_r
\end{equation}

Now the original formulation of the problem stated in \ref{eq:userpositionfull}
can be rewritten in linear form as
\begin{equation}
  \begin{array}{l}
    \Delta \rho_1 = a_{x1} \Delta x_r + a_{y1} \Delta y_r + a_{z1} \Delta z_r - c \delta_r \\
    \Delta \rho_2 = a_{x2} \Delta x_r + a_{y2} \Delta y_r + a_{z2} \Delta z_r - c \delta_r \\
    \Delta \rho_3 = a_{x3} \Delta x_r + a_{y3} \Delta y_r + a_{z3} \Delta z_r - c \delta_r \\
    \Delta \rho_4 = a_{x4} \Delta x_r + a_{y4} \Delta y_r + a_{z4} \Delta z_r - c \delta_r
  \end{array}
\end{equation}

These equations can then be put in matrix form; by defining
\[
  \begin{array}{l}
  \Delta \bm{\rho} = \begin{bmatrix}
    \Delta \rho_1 \\
    \Delta \rho_2 \\
    \Delta \rho_3 \\
    \Delta \rho_4
  \end{bmatrix}

  \>

  \bm{H} = \begin{bmatrix}
    a_{x1} & a_{y1} & a_{z1} & 1 \\
    a_{x2} & a_{y2} & a_{z2} & 1 \\
    a_{x3} & a_{y3} & a_{z3} & 1 \\
    a_{x4} & a_{y4} & a_{z4} & 1 \\
  \end{bmatrix}

  \>

  \Delta \bm{x} = \begin{bmatrix}
    \Delta x_r \\
    \Delta y_r \\
    \Delta z_r \\
    -c \Delta \delta_r
  \end{bmatrix}
  \end{array}
\]
we can then write
\begin{equation}
  \Delta \bm{\rho} = \bm{H} \Delta \bm{x}
\end{equation}
which has solution
\begin{equation}
  \Delta \bm{x} = \bm{H}^{-1} \Delta \bm{\rho}
\end{equation}

Once the unknown are computed, the actual user position can be computed using
\ref{eq:incrementals}. This process works well as long as the displacement from
the approximate position is small enough. However, subsequent iterations of the
method will improve the accuracy of the result.

In reality though, the true user-to-satellite range measurements are affected by
independent errors such as measurement noise and multipath. Such errors
constitute a component that affects the calculated user position. To minimize
these errors, one technique is to perform measurements to more than four
satellites, using least square estimation methods to minimize the effect of the
independent source of errors. This is a feature that most of today's user
receivers employ.

\section{Determining user's velocity}
Another component of the PVT solution is velocity. Satellite navigation provides
the capability of measuring a receiver's three-dimensional velocity, here
denoted as $\bm{\dot u}$. This is generally determined using one of two
techniques: either as an approximate derivative of the receiver position, taking
the difference of two subsequent measurements and dividing it by the time
elapsed between the two. This technique can provide satisfactory results if the
receiver's velocity is nearly constant and if the errors in determining the two
positions are small enough. For this reason most modern receivers use a
technique based on measurement of the carrier frequency and a precise estimation
of the received Doppler frequency.

\par

Generally speaking, the received frequency at the receiver's antenna can be
approximated by the classical Doppler equation:
\begin{equation} \label{eq:doppler}
  f_R = f_T \left(1-\frac{\bm{v_r} \cdot \bm{a}}{c}\right)
\end{equation}
where $f_T$ is the transmitted frequency at the satellite's end, $f_R$ is the
received frequency at the receiver's antenna, $c$ is the speed of light,
$\bm{a}$ is the unit vector directed along the line of sight between satellite
and receiver, and $\bm{v_r}$ is the satellite-to-receiver relative velocity
vector. Given $\bm{v}$ as the satellite's velocity, we can express $\bm{v_r}$ as
\begin{equation} \label{eq:user_velocity}
  \bm{v_r} = \bm{v} - \bm{\dot u}
\end{equation}
This relative motion results in a Doppler offset that can be expressed by
\begin{equation}
  \Delta f = f_R - f_T = -f_T \frac{(\bm{v} - \bm{\dot u}) \cdot \bm{a}}{c}
\end{equation}

At this point several techniques can be used to determine $\bm{\dot u}$ from the
received Doppler frequency. The one described here assumes that the user
position has already been determined and computes, together with the receiver's
velocity $\bm{\dot u} = (\dot x_u, \dot y_u, \dot z_u)$, also the clock drift
$\dot t_u$.

We start by substituting \ref{eq:user_velocity} into \ref{eq:doppler}, which
yields
\begin{equation} \label{eq:doppler2}
  f_{Rj} = f_{Tj}\left\{ 1-\frac{1}{c}\left[ \left( \bm{v}_j - \bm{\dot u} \right) \cdot \bm{a}_j \right] \right\} 
\end{equation} 

At this point the receiver needs to estimate the received signal frequency. This
measurement is also subject to error due to the bias induced by drifting of the
local clock, expressed by $\dot t_u$. Hence, being $f_j$ the estimate of the
received frequency, we can put clock drift, actual and estimated received
frequency in the following relation
\begin{equation} \label{eq:received_freq}
  f_{Rj} = f_j (1 + \dot t_u)
\end{equation}

Substituting \ref{eq:received_freq} into \ref{eq:doppler2} and performing some
basic algebraic manipulation we obtain
\begin{equation}
  \frac{c(f_j - f_{Tj})}{f_{Tj}}+\bm{v}_j \cdot \bm{a}_j = \bm{\dot u} \cdot \bm{a}_j - \frac{c f_j \dot t_u}{f_{Tj}}
\end{equation}
if then we expand the dot products using the three-dimensional components of the
vectors we reach
\begin{equation}
  \frac{c(f_j - f_{Tj})}{f_{Tj}} + v_{xj}a_{xj} + v_{yj}a_{yj} + v_{zj}a_{zj} =
  \dot{x_u} a_{xj} + \dot{y_u} a_{yj} + \dot{z_u} a_{zj} - \frac{c f_j \dot t_u}{f_{Tj}}
\end{equation}

In this expression all variables on the left side are either calculated or
derived from measured values. To make the equation more compact we can then
introduce a new variable
\begin{equation}
  d_j = \frac{c(f_j - f_{Tj})}{f_{Tj}} + v_{xj}a_{xj} + v_{yj}a_{yj} + v_{zj}a_{zj}
\end{equation}

Additionally, the term $f_j/f_{tj}$ is tipically very close to 1 and the
resulting error is negligible. With this further simplification we obtain
\begin{equation}
  d_j = \dot{x_u} a_{xj} + \dot{y_u} a_{yj} + \dot{z_u} a_{zj} - c \dot t_u
\end{equation}

Now the equation contains four unknowns which can be solved by performing four
distinct measurements against four different satellites, yielding a system of
equations that can represented and solved with the following matrices:
\[
  \begin{array}{l}
    \bm{d} = \begin{bmatrix}
    d_1 \\ d_2 \\ d_3 \\ d_4
  \end{bmatrix}

  \>

  \bm{H} = \begin{bmatrix}
    a_{x1} & a_{y1} & a_{z1} & 1 \\
    a_{x2} & a_{y2} & a_{z2} & 1 \\
    a_{x3} & a_{y3} & a_{z3} & 1 \\
    a_{x4} & a_{y4} & a_{z4} & 1 \\
  \end{bmatrix}

  \>

  \bm{g} = \begin{bmatrix}
    \dot x_u \\
    \dot y_u \\
    \dot z_u \\
    -c \dot t_u
  \end{bmatrix}
  \end{array}
\]
which then corresponds to
\begin{equation}
  \bm{d} = \bm{H} \bm{g}
\end{equation}
so the solution for the user's velocity and time drift is
\begin{equation}
  \bm{g} = \bm{H}^{-1} \bm{d}
\end{equation}

As in the previous case, this solution is affected by errors propagating across
the calculation. Also similar to the previous case, it's possible to perform
measurements against more than four satellites and use least square estimates to
increase the accuracy.

\section{Terrestrial reference frame}
So far in the discussion of how to resolve user's PVT equation we took for
granted that a coordinate reference frame was provided. We'll now look into how
this coordinate reference frame is modeled and how receivers can map between
this reference frame and a pair of longitude and latitude estimates.

\par

In general, two kind of models can be used: inertial and rotating. When
describing the orbit of satellites, it's more useful to use an intertial system;
in particular, an Earth Centered Inertial (ECI) system has origin at the center
of mass of the Earth and axes that point in fixed directions with respect to the
stars.

When dealing with measuring and describing user position, instead, it's more
useful to use a reference system that's fixed with respect to the Earth (i.e. a
reference system that rotates with it). Such a system is called Earth Centered
Earth Fixed (ECEF). In such a system the xy-plane is coincident with Earth's
equatorial plane: the x axis pointing to 0° longitude and the y axis pointing to
90°E longitude. The z axis is chosen to be normal to the equatorial plane in the
direction of the geographical North Pole, thus forming a right-handed coordinate
system.

These frames are useful for describing and solving the PVT problem, but are less
so when a fix needs to be presented to the user. This normally is better
described by a pair of longitude and latitude measurements. Therefore in order
to make this transformation, a physical model of the Earth is needed. Such model
is not unique and varies between satellite systems; Galileo uses what's known as
GTRF (Galileo Terrestrial Reference Frame).

The GTRF is an independent realization of an international standard called ITRS
(International Terrestrial Reference System) which defines a framework to
describe geodetic reference frames. This framework is used to produce periodic
updates to the ITRF; by Galileo system requirement the three-dimensional
difference between a position in GTRF and the most recent ITRF should not exceed
3cm.

The purpose of such a reference frame is to define the Earth's surface by a
closed equation representing an ellipsoid with a certain number of degrees of
freedom. These degrees of freedom are then represented by parameters that can be
sent along the navigation message and subsequently used to resolve user's
position in terms of longitude, latitude and height with respect to this
reference frame; defined in this manner, these parameters are called
\textit{geodetic}.

Given a user's position vector $\bm{u} = (x_u, y_u, z_u)$ in ECEF coordinates,
we can compute the user's geodetic longitude $\lambda$ as the angle between the
user and the x axis measured in the xy-plane:
\begin{equation}
  \lambda = \begin{cases}
    \arctan \left(\frac{y_u}{x_u}\right),& \text{if } x_u \geq 0 \\
    180°+\arctan \left(\frac{y_u}{x_u}\right),& \text{if } x_u < 0 \text{ and } y_u \geq 0\\
    -180°+\arctan \left(\frac{y_u}{x_u}\right),& \text{if } x_u < 0 \text{ and } y_u < 0\\
  \end{cases}
\end{equation}

Geodetic latitude $\phi$ and height $h$ are defined in terms of the normal to
the ellipsoid from the user's position $\bm{n}$. Geodetic height is defined as
the minimum distance of the user to the surface of the reference ellipsoid.
Geodetic latitude is the angle between the normal $\bm{n}$ and its projection
into the equatorial plane. Conventionally $\phi$ is set to be $> 0$ if the user
is in the northern empisphere ($z_u > 0$) and $< 0$ if the user is in the
southern emisphere ($z_u < 0$).

To calculate $\phi$ and $h$ several methods exist, both closed-form and
iterative; one example of the latter is a highly convergent method by Bowring
which uses four parameters in the description of the ellipsoid:
\begin{itemize}
\item $a$ semimajor axis, i.e. radius of the cross-section taken at the
  equatorial plane
\item $b$ semiminor axis, i.e. radius of the polar section of the Earth
\item $e$ eccentricity, defined as 
  \[
    e = \sqrt{1 - \frac{b^2}{a^2}}
  \]
\item $e'$ second eccentricity, defined as
  \[
    e' = \sqrt{\frac{a^2}{b^2} - 1} = \frac{a}{b}e
  \]
\end{itemize}
The resolution method, given these parameters, is as follows:
\[
  \begin{array}{l}
    p = \sqrt{x^2 + y^2} \\
    \tan u = \big(\frac{z}{p}\big)\big(\frac{a}{b}\big) \\
    \text{iteration loop} \\
    \cos^2 u = \frac{1}{1+\tan^2 u} \\
    \sin^2 u = 1 - \cos^2 u \\
    \tan \phi = \frac{z + e'^2 b \sin^3 u}{p - e^2 a \cos^3 u} \\
    \tan u = \frac{b}{a}\tan \phi \\
    \text{until $\tan u$ converges, then} \\
    N = \frac{a}{\sqrt{1 - e^2 \sin^2 \phi}} \\
    \text{if $\phi \neq \pm 90 \degree $} \\
    h = \frac{p}{\cos \phi} - N \\
    \text{otherwise if $\phi \neq 0$} \\
    h = \frac{z}{\sin \phi} - N + e^2N
  \end{array}
\]

\section{Additional sources of errors}
So far we considered as a potential source of error just the clock drift at the
receiver's end, provided the drift at the satellites' end is corrected by the
ground segment and the correction parameters transmitted with the navigation
message. The reason for this is that this source of error has an impact that
makes any other error less than significant; nevertheless, once clock drift has
been accounted for, other sources of error can become relevant and affect the
accuracy of the final fix.

These error sources include:
\begin{itemize}
  \item Ephemeris errors: these happen as the ephemerides transmitted by ground
    mission are a prediction of the actual trajectory. In the case of Galileo,
    differences between the actual orbit path and the broadcasted one can
    introduce errors in pseudorange measurements in the order of 1.6m $\pm$
    0.3m.
  \item Atmospheric errors: these are of different kind. The more general one is
    due to the dispersion effect that happens in mediums with an index of
    refraction that is a function of the wave frequency (as is in the case of
    Earth's atmosphere). This results in different propagation speeds between
    signal phase and information. Another kind of error is due to the crossing
    of the signal of Earth's ionosphere; the effect is similar to the general
    atmospheric propagation delay, but its magnitude is strong enough that
    correction parameters are provided as part of the navigation message.
    Since this error is frequency-dependent, another way to compensate for this
    effect is to use a dual-frequency receiver, making measurements on two
    separate frequency bands.
  \item Relativistic effects: the satellites' clocks are affected by both
    general relativity and special relativity laws. Galileo OS SIS ICD doesn't
    make a specific statement with reference to relativistic effects, but
    provides nevertheless four clock correction parameters to compensate for
    various aspects that might influence the frequency of the onboard clocks.
\end{itemize}

\section{Signal acquisition and tracking through PRN codes}
% TODO
% - Doppler effect?
% - difference between acquisition and tracking
% - acquisition through autocorrelation functions
% - tracking through PLL
