%!TEX root = ../dissertation.tex

\chapter{A brief history of GNSS}

% toc
%  - gps (3)
%    - history
%    - structure
%  - galileo (3)
%  - other constellations (2)
%  - interoperability challenges (1)
%  - future vision (1)

In order to understand the context in which the problem of data authentication
for Galileo is studied, it's useful to know how the major GNSS constellations
evolved and what are the traits that characterize them at present time. In
this chapter we'll briefly introduce Galileo, GPS, and the other main players
in the satellite positioning arena.

\section{Global Positioning System (GPS)}
In February 1978, the first satellite of the Block I family was put in orbit by
the U.S. government. It was the first of a series of 11 satellites that will
constitute the prototype of what's currently known as GPS. This was the first
concrete realization of what would soon be world's most used radio navigation
system, and was the result of putting together ideas and prototypes that had
been developed by several U.S. military branches.

\subsection{GPS overview}
As of today, GPS is comprised of three segments: the space segment, ground
segment and user segment.

The space segment is the constellation of satellites that broadcast the radio
signal used by users to determine their positions. The ground segment is the
operating network in charge of maintaining satellites in their proper orbit and
of making sure the data that's sent is accurate. The user segment is composed of
user receiving equipment. GPS receivers are mainly responsible for resolving the
PVT equation after having acquired and tracked a sufficient number of
satellites. In addition to that, GPS receivers can also offer other kinds of
measurements - like platform attitude (heading, pitching and roll) - or function
as precise sources of time.

\subsection{Ground segment}
Fundamental to the correct operation of GPS is the health of the satellite
constellation and the accuracy of the navigation message. These two tasks are
the main responsibility of the ground segment - which includes a redundant
Master Control Station (MCS), monitoring stations (MS) and ground antennas (GA).
The MCS performs various functions like generating the navigation message,
monitoring SVs health and performing housekeeping operations on them, and
maintaining time synchronization against UTC. 

In order to do so, the MCS communicates through satellites via GAs, which are
deployed across the globe so that each satellite can be reached by one to four
antennas depending on its position. A GA is a full-duplex S-band communication
facility that can hold a dedicated control and command session with a SV,
operating under control of the MCS.

To receive precise information over the constellation's health, the MCS also
operates a set of MSs that is capable of receiving GPS signal over the full
L-band and to send this signal together with meteorological data back to the
MCS. In addition to that, MSs include as part of their equipment high precision
redunant cesium clocks, which are used as a reference to measure precision of
the received data.

\subsection{Space segment}
The original design of GPS includes 24 satellites, divided in six orbital
planes positioned approximately 26,600 km from the mass center of Earth. The
nominal orbital period is one-half of a sidereal day, or 11 hours and 58
minutes. This design allows for global coverage, meaning that at nominal
constellation operativity a minimum of three satellites is visible at any
position on Earth's surface.

Space vehicles (SVs) had undergone several modernizations steps starting from
the first launch in 1973. Subsequent developments are organized into blocks,
Block I being the first. The current constellation includes SVs from blocks IIR,
IIR-M and IIF, with a new block IIIA being planned for the next years.

Nevertheless, the main components of a SV remain the same: an antenna farm,
antennas for communication with the ground segments, a solar array for providing
energy to the satellite, a propulsion engine for orbital adjustments,
high-precision rubidium clocks, memory storage for robust operation under loss
of link with the ground segment, and of course a computing unit. Each of these
main building blocks are subject of the modernization efforts, with the goal of
making each new block more robust to adverse oribting conditions, more tolerant
to failures, more flexible to changing requirements, and more precise. At the
same time, every improvement contributes to achieve higher SV longevity: as an
example the design life of a Block I satellite was 5 years, while that of a
Block IIF satellite is 12 years.

\subsection{GPS signals features}
Modernization of satellites doesn't happen only because of hardware
improvements, but also because the GPS specification is in constant evolution.
The current GPS signal plan comprises transmission on three separate bands: L1,
L2 and L5. The former two bands are part of the original plan, which included a
C/A (coarse acquisition) signal on L1 and a military signal named P(Y) on both
L1 and L2. Modernization of the signal plan introduced three new civil signals,
one on L1 (L1C), one on L2 (L2C) and one on a new band L5 which will also
support SoL (Safety of Life) operations. In addition to this, a new military
signal (M) has been added on the L1 and L2 bands.

Originally, the GPS signal used a BPSK modulation that in conjunction with the
use of spreading code allowed for maximization of bandwidth usage. To accomodate
for transmission of different signals on the same frequency band, BOC modulation
has been introduced in newer signals. This achieves sufficient spectral
separation for existing receivers to keep working and for new receivers to be
able to isolate each transmission. This idea has been also a major driver in the
Galileo's signal plan as explained later in this chapter.

\section{History of Galileo}

\section{Other constellations}

\section{Interoperability challenges}

\section{The future of GNSS}
