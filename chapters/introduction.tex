%!TEX root = ../dissertation.tex
\chapter{Introduction}
\label{introduction}

%- context
%  - other systems, introduce Galileo
%  - evolution in the space vehicles technology
%  - evolution of the technology itself:
%    - increase in performances by aided positioning (GBAS, WiFi)
%    - GPS signal evolution
%- statement of the problem
%  - with the current state-of-art there is no security for civil clients
%  - only secure transmission is for military purposes, uses encryption which
%cannot work for an open signal
%- statement of the response
%  - intense research in the past years
%  - ESA selected one algorithm: TESLA
%  - what guarantees TESLA provides
%  - there are no implementations that make for easy tuning of parameters
%  - a simulator for the whole Galileo signal spectrum is being developed at DEI
%  - the simulator can also help test the configuration and operation of TESLA
%- roadmap of the thesis

Since its introduction in 1973, GPS has seen tremendous growth in adoption.
Thanks to the proliferation of use cases, positioning technologies have become
the basis upon which many services are built. One of the most obvious ones is of
course route guidance, the most important one arguably being aircrew assistance:
GPS has played a central role in the steady improvement of critical aircraft
operation reliability, such as landing and take off. A fast-growing field of
application is also agriculture, which benefits from the high precision of
satellite positioning by automating mechanical operations on large fields like
harvesting and sowing - in many cases tractors are today totally unmanned,
relying solely on the precision of GPS for covering the field of operation with
high efficiency.\\
But the real explosion of GPS arguably started when major manufacturer started
to embed receivers in smartphone and tablets. As of today, it's given for
granted that any sufficiently evolved digital device is capable of decoding a
GPS signal and estimate its position. This opened the door to a complete new set
of applications of which the full span has yet to be seen. From routing
applications to local recommendations engine, none of them would have much
impact without the possibility of locating the user offered by GPS.\
A new field of application that's capturing the interest of major companies is
the automotive one: whether self-driving car will be soon a common reality, or
whether safety concerns and regulations will push this reality far in the
future, it's undeniable that GPS plays a central role in the definition of this
new reality.

Following the success of the American experiment, other global powers decided to
build their own version of GPS. As of today Russia, China, European Union and
Japan operate their own satellite constellation for positioning purposes - even
if each to a different degree of platform maturity, stability, scale and reach.
