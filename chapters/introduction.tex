%!TEX root = ../dissertation.tex
\chapter{Introduction}
\label{introduction}

% TODO
% - roadmap of the thesis

Since its introduction in 1973, GPS has seen tremendous growth in adoption.
Thanks to the proliferation of use cases, positioning technologies have become
the basis upon which many services are built. One of the most obvious ones is of
course route guidance, the most important one arguably being aircrew assistance:
GPS has played a central role in the steady improvement of critical aircraft
operation reliability, such as landing and take off. A fast-growing field of
application is also agriculture, which benefits from the high precision of
satellite positioning by automating mechanical operations on large fields like
harvesting and sowing - in many cases tractors are today totally unmanned,
relying solely on the precision of GPS for covering the field of operation with
high efficiency.\\
But the real explosion of GPS arguably started when major manufacturer started
to embed receivers in smartphone and tablets. As of today, it's given for
granted that any sufficiently evolved digital device is capable of decoding a
GPS signal and estimate its position. This opened the door to a complete new set
of applications of which the full span has yet to be seen. From routing
applications to local recommendations engine, none of them would have much
impact without the possibility of locating the user offered by GPS.\
A new field of application that's capturing the interest of major companies is
the automotive one: whether self-driving car will be soon a common reality, or
whether safety concerns and regulations will push this reality far in the
future, it's undeniable that GPS plays a central role in the definition of this
new reality.

Following the success of the American experiment, other global powers decided to
build their own version of GPS. As of today Russia, China, European Union, India
and Japan operate their own satellite constellation for positioning purposes -
even if each to a different degree of platform maturity, stability, scale and
reach.

Russian GLONASS is the oldest of these newer constellations, and reaching a
precision close to that of GPS. The Chinese BeiDou, Indian IRNSS and Japanese
QZSS are all considered as of now regional systems, meaning they only cover a
limited area around their respective country of origin.

Possibly the most promising development is provided by the European Galileo
constellation, the youngest of the systems mentioned so far. With a nominal
accuracy of 1 meter for civil signal it promises to outperform the current GPS
capabilities. This constellation is currently operational with 14 satellites,
with a planned constellation of 30 space vehicles; at the same time, research is
still being held under many aspects of the system.

No matter the size and age of these systems, overall satellite positioning
technology has seen various steps of improvements over the years, and in two
main areas: hardware improvements on the space vehicles and optimizations in the
usage of the allocated radio frequencies. GPS space vehicles have been
continuously modernized, resulting in the release of 7 different satellite
versions (blocks I to IIIA), and another 2 planned but not yet scheduled (IIIB
and IIIC). Improvements generally range across improvements of nominal life,
increases in the radio transmission performance (e.g. signal power at Earth's
surface) and hardware addition to better cope with errors (e.g. addition of
laser retro-reflectors in block IIIA to disentangle clock errors from ephemeris
errors). \\ 
Another reason for the development of new space vehicles is to allow
deployment of new signals. For example, the last planned block of GPS satellites
is going to introduce 3 new signals (L1C, L2C and M). Each of them improves
on the existing signals allowing for increased robustness against interference
and data loss, and higher accuracy through increased bandwidth, longer spreading
codes and higher power.

Together with this, recent developments have also happened in the direction of
enhancing satellite positioning by using other, more local sources. For example,
aviation has stringent needs in terms of monitoring and client warning in case
of, e.g., clock problems. In such cases aircrafts must be notified in real-time
that they can't rely on a GNSS for precision approach. Neither GPS nor GLONASS
support this use case, so a support system was developed under the name of GBAS
(Ground-Based Augmentation System). Aside from this, a cluster of other
augmentation systems with similar purposes have been deployed; such systems go
under the name of SBAS (Satellite-Based Augmentation System).\\
More recently, smartphones have begun to support augmentation of the satellite
signal through both WiFi and cell tower triangulation, resulting in better user
experiences in terms of faster and more precise resolution of the user's
position.

Despite all these efforts to improve on the original GPS signal, little to no
practical development has so far been made in making the civil signal secure.
Current signals position themselves on two very far end of the security
spectrum. On one side, military signals provide security through encryption;
this not only provides state-of-the-art security to the signal, but also allows
for restricted access to it: only receivers with authorized access are able to
decode the signal. On the other side of the spectrum, use of civil signals is by
nature open to everyone, but as of today there is no mechanism to guarantee the
authenticity of the signal itself. Malicious users can choose from a wide range
of spoofing and replay attacks, allowing for almost undisturbed disruption of
service. While letting a user think they're located in a remote island while
being in the middle of a most crowded city might appear like a practical joke,
even a subtle change in the position or velocity of an aircraft might have
catastrophic consequences. Even without putting lives at risk, attacks that aim
at drifting a receiver's clock can be a component of more complex network
attacks - as many security algorithms rely on coarsely precise clocks for
avoiding replay attacks. If nothing else, a whole class of denial of service
attacks can be conceived exploiting the vulnerability of today's civil
positioning technologies.

In recent years this topic gained a strong momentum, and research on different
levels and different directions has been performed to solve the problem of
securing GNSS signals. Proposed solutions can be generally categorized based
on what part of the GNSS signal they aim to protect. A number of proposals have
been developed to make the spreading codes more robust, to prevent from, e.g.,
selective-delay attacks. These techniques seem to provide better overall
security but are arguably more demanding on the receiver side in terms of
implementation. A simpler general approach is instead that of leaving the
overall signal structure unchanged, protecting only the navigation message
carried by the signal. It is this last approach that has been currently taken to
protect Galileo's public regulated service in the form of the TESLA protocol.

The TESLA protocol is an asymmetric data authentication protocol specifically
designed to work over a continuous, real-time stream of data over a lossy
channel. The main idea behind it is that of using hash chains to generate
message authentication codes, with keys being revealed periodically to allow for
verification of previously-received messages. In a recent proposal, the
theoretical protocol has been adapted to fit the current Galileo data signal by
taking advantage of the reserved bit blocks factored into the actual frame
structure, which allows for support over the E1b and E1c signals. 

As of today, the protocol remains in form of proposal, but in the course of 2018
it's supposed to be part of the Signal in Space (SIS) testing. The goal of this
publication is to provide a software implementation of this scheme that follows
the guidelines approved by ESA, to allow for easy but accurate testing of the
operational parameters that are going to be subject to change and/or remain
variable in the final implementation. This effort joins forces with the
more-GOSSIP project being developed at the Department of Electrical Engineering
of the Università degli Studi di Padova. This project aims at developing a
software simulator that can test a number of possible attacks on Galileo (and
GPS) signals, and implementation of the proposed Galileo OSNMA (Open Service
Navigation Message Authentication) is part of it.
