%!TEX root = ../dissertation.tex

\chapter{Data authentication in Galileo}

So far we've described the classic radio satellite navigation problem and
solution. This generally accepted model works well but provides no security
guarantees. In particular, it doesn't prevent malicious users to forge ad-hoc
signals and impersonate a satellite, or even a whole GNSS.

In this chapter we'll describe the general data authentication problem for open
signals in GNSS environments, the threat model and the possible attacks. We will
then move on to describe TESLA, a data authentication protocol suited for
streaming, lossy channels that the Galileo program chose to implement in order
to secure the navigation message. We'll also describe the proposed
implementation of TESLA in the current Galileo subframe structure.

% TOC
% - threat model
% - possible attacks
% - TESLA
% - ICD proposal to integrate TESLA in the current Galileo navigation message

% TESLA wants
% - local clocks of the sender and receiver don't drift too much during a
%   session -> client needs to know a loose upper bound on the maximum
%   synchronization error
% - in scheme IV at session setup the sender should send and authenticate the
%   starting time of the first interval, the duration of an interval and the
%   maximum delay d. The specs don't do this in full
% - in scheme IV it's required that the sender sends along the interval index i,
%   the specs don't do this?
% - the specs don't specify any value for the propagation delay. It would be
%   interesting to provide a recommended value
% - suggest that receivers initially synchronize time over the network before
%   starting to perform authentication with TESLA due to the protocol's
%   requirements? -> that would still be somehow insufficient, as it requires
%   the network time sync protocol is also authenticated
%     Or maybe the receiver can estimate an upper bound to the synchronization
%   error during the acquisition phase?
% - a possible DOS attack is to forge a packet marked as being from an interval
%   far in the future. A possible fix is to ignore packets if they could have
%   not been sent yet -> does a Galileo receiver have any means to do this? Any
%   condition on the fields that guarantees that this cannot happen?
