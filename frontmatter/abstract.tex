%!TEX root = ../dissertation.tex
% Abstract

Since its first incarnation in the 70s under the "NAVSTAR" name, satellite
positioning acquired more and more relevance in the life of everyone. Directly
or indirectly, we all rely on technological aids to precisely define our
position. This exponential trend uncovered in recent years the need to secure
the information transmitted by legitimate satellites: military signals are
encrypted before transmission with schemes that allow only an authorized and
restricted set of users, but for the civil case such a restriction clashes with
the idea of having an open signal accessible by anyone. Yet, leaving the
transmitted data unprotected opens the door to various kinds of attack.

\vspace{\baselineskip}

It is with the introduction of a satellite constellation operated by the
European Union under the name of Galileo that the topic of data authentication
for open Global Navigation Satellite Systems (GNSS) signal took a practical
turn. Some authentication schemes have been developed on paper to suit the low
data rate available, resulting in the European Commission releasing an update to
Galileo's Interface Control Document (ICD) to include an implementation of
Timed Efficient Stream Loss-tolerant Authentication (TESLA), a protocol suited
to authenticate streams of data that makes no assumption on the time when a
receiver starts receiving the radio signal.

In this paper a critical analysis of the proposed specification is performed, to
understand the implications on the receiver side in terms of performance and
resource consumption.
